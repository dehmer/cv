\documentclass[12pt,a4paper,sans]{moderncv}
\definecolor{color1}{rgb}{0.149, 0.3804, 0.6118} %% #26619c - lapis lazuli
\moderncvstyle[]{classic}

\usepackage[utf8]{inputenc}
\usepackage[T1]{fontenc}
\usepackage[ngerman]{babel}
\usepackage[scale=0.78]{geometry}
\usepackage{lmodern}

\newcommand{\tech}[1]{\par $\triangleright$ #1}

\firstname{Horst}
\familyname{Dehmer}
\title{CV}
\address{Simmeringer Hauptstraße 48/2/48}{1110 Wien}{Österreich}
\mobile{+43~(664)~544~1826}
\email{horst.dehmer@gmail.com}
\social[github]{github.com/dehmer}

\quote{Controlling complexity is the essence of computer programming. --- Brian W. Kernighan (1976)}

\begin{document}
\makecvtitle

\section{Persönliche Informationen}
\cvitem{Staatsbürger\-schaft}{Deutsch}
\cvitem{Geburtsdatum, --Ort}{11. August 1969, Darmstadt, Deutschland}

\section{Ausbildung}
\cventry
{1989--1999}
{Diplom-Informatik}
{Technische Universität}
{Darmstadt, Deutschland}
{}
{Nebenfach: Arbeits--, Betriebs-- und Organisationspsychologie}

\section{Berufliche Erfahrungen}
\cventry
{2014--2025}
{Mitbegründer, Geschäftsführender Gesellschafter}
{}
{Syncpoint GmbH}
{Wien, Österreich}
{
  Kunden: Salzburg AG,
  Bundesministerium für Landesverteidigung - BMLV/ÖBH,
  Bundesministerium für Bildung, Wissenschaft und Forschung - BMBWF,
  Austrian Institute of Technology - AIT
}

\cventry
{2023--2025}
{Forschung}
{Unterstützungssoftware für den taktischen Aufklärungsverbund - GUSTAV}
{Bundesministerium für Landesverteidigung}
{Österreich}
{
  Entwicklung und Bereitstellung eines integrierten Systemprototypen mit
  \emph{Technology Readiness Level} - TRL 7 (\emph{Technology Demonstration}) für
  die FORTE-Vorgängerprojekte PIONEER und BOOST.
  Digitalisierung des taktischen Aufklärungsverbunds innerhalb des NATO Intelligence Cycle
  mit durchgängiger Prozessunterstützung für die ``All-Source Intelligence Cell''.
  Fokus auf Erfassung, Strukturierung und Integration von Aufklärungsergebnissen, sowie die periodische Erstellung
  von Berichten inkl. Lagekarten mit geo-referenzierter Darstellung von Ereignissen.
  \tech{
    React, Svelte, OpenLayers, PostgreSQL, Node.js, Python, Llama 3.1,
    Answer Set Programming - ASP, MinIO, RDF/OWL, Apache Airflow/Celery, RabbitMQ, Oxigraph/SPARQL,
    Puppeteer
  }
}

\cventry
{2023--2024}
{Forschung}
{Adaptable Automated Intelligence Gathering Processes for Decision Support - BOOST}
{
  Bundesministerium für Landesverteidigung
  in Kooperation mit Austrian Institute of Technology - AIT,
  finanziert durch Forschungsförderungsgesellschaft mbH - FFG
}
{Österreich}
{
  Unterstützung für Analysten durch die Automatisierung von ausgewählten Schritten des
  Aufklärung-Lebenszyklus: Strukturierung von gesammelten Aufklärungsergebnissen, Extraktion von
  relevanten Informationen, Validierung von Hypothesen und Ableiten von Handlungsempfehlungen
  für weitere Aktionen.
  \tech{
    React, Svelte, PostgreSQL, Node.js, Python, Llama 3.1, Answer Set Programming - ASP,
    MinIO, RDF/OWL, Apache Airflow/Celery, RabbitMQ
  }
}

\cventry
{2022}
{Beratung, Training, Entwicklung}
{Friendly Force Information System -- Teil 2}
{Österreichisches Bundesheer (IKT\&CySihZ)}
{Wien, Österreich}
{
  Erweiterungen des Friendly Force Information Systems im Web-GUI und Backend.
  Abbau von \emph{Knowledge Silos}: Technische Weiterbildung und Training der
  hausinternen Entwickler.
  \tech{ActiveMQ, JMS, Akka Actors und Akka Persistence, PostgreSQL, Event Sourcing}
}

\cventry
{2022--2025}
{Architektur, Entwicklung/Backend}
{PV-Metering Portal}
{Salzburg AG}
{Salzburg, Österreich}
{
  Periodische Abfrage der Energiezähler der Photovoltaik-Anlagen von verschiedenen
  Herstellern (Siemens, Janitza, Carlo Gavazzi) über Modbus/TCP.
  \tech{PostgreSQL, PL/pgSQL, Node.js, Modbus/TCP, Event Sourcing (JSON)}
}

\cventry
{2022--2023}
{Forschung}
{Kommunikations-, Sicherheits- und Informationsübertragungssystem für die aufgabenspezifische Lagedarstellung und Einsatzführung - KOSINUS}
{Bundesministerium für Landesverteidigung}
{Österreich}
{
  \tech{Matrix, Element Desktop- und Android-Clients, Docker, JavaScript, Android, Kotlin}
}

\cventry
{2020}
{Architektur, Entwicklung/Backend}
{COVID-19 Meldesystem}
{Bundesministerium für Bildung, Wissenschaft und Forschung}
{Österreich}
{
  Client/Server-Anwendung für Universitäten, Schulen und andere
  Bildungseinrichtungen. Meldung und Nachverfolgung von Infektionen
  und Genesungen von Angestellte und Studierende.
  \tech{Node.js, React, Express, REST, PostgreSQL, Event Sourcing (JSON)}
}

\cventry
{2020--2022}
{Forschung}
{Interoperability and Digitization of Intelligence Gathering Processes - PIONEER}
{
  Bundesministerium für Landesverteidigung
  in Kooperation mit Austrian Institute of Technology - AIT,
  finanziert durch Forschungsförderungsgesellschaft mbH - FFG}
{Österreich}
{
  Vorgänger für das BOOST Projekt mit Fokus auf die automatische und manuelle
  Strukturierung von gesammelten Aufklärungsergebnissen, AI-gestützte Integration
  von Informationen in eine Wissensdatenbank und das Bereitstellen von dedizierten
  Benutzeroberflächen für die Präsentation von Informationen in verschiedenen
  zeitbasierten und georeferenzierten Sichten.
  \tech{React, spaCy Natural Language Processing, Node.js, LevelDB, Python, RDF, Docker, MinIO}
}

\cventry
{2018--2019}
{Architektur, Entwicklung}
{Gasportal}
{Salzburg AG}
{Salzburg, Österreich}
{
  Abfrage der Zählerstände und anderer Parameter von Gastankstellen
  zum Zweck der Verrechnung sowie der Planung und Durchführung von Wartungsarbeiten.
  \tech{PostgreSQL, PL/pgSQL, Node.js, Modbus/TCP, Event Sourcing (JSON)}
}

\cventry
{2016--2017}
{Architektur, Entwicklung}
{Friendly Force Information System -- Teil 1}
{Österreichisches Bundesheer - Führungsunterstützungszentrale (FüUZ)}
{Wien, Österreich}
{
  Beratung, Design und Umsetzung eines Friendly Force Information Systems.\newline
  Überarbeitung der Infrastruktur und der Werkzeuge für die Entwicklung.
  Einführen von Virtualisierung für die Entwicklungsserver (Maven Repository, Jenkins)
  und Modernisierung der Werkzeuge, u.a. git, IntelliJ, SonarQube und Java 8.\newline
  Event-getriebene Architektur basierend auf Microservices zum Verfolgen und Anzeigen
  von GPS Positionen eigener Einheiten. Unterstützung für taktischen Truppenfunk
  (\emph{Combat Net Radio} - CONRAD) und BOS-Funk (Behörden und Organisationen
  mit Sicherheitsaufgaben.)
  \tech{Java, Akka Actor Framework, ActiveMQ, PostgreSQL}
}

\cventry
{2014--2016}
{Architektur, Entwicklung}
{Communication Gateway - CommGW}
{Salzburg AG}
{Salzburg, Österreich}
{
  Zentralisierte Überwachung von, und Kommunikation zwischen, geschäftskritischen
  Anwendungen und Prozessen basierend auf einer Microservice-Architektur.
  Kundenspezifische Workflow-Engine für eine hohe Flexibilität; Nachrichtenversand
  mit garantierter Zustellung.
  \tech{Node.js, Docker, RabbitMQ, MongoDB, Express, SMTP, AMQP, FTP, REST, IEC 60870-5-101, IEC 60870-5-104}
}

\cventry
{2011--2014}
{Entwicklung/Backend}
{}
{Frequentis AG}
{Wien, Österreich}
{
  Entwurf und Entwicklung von verschiedenen Schnittstellen zu Avitech Aeronautical
  Exchange Layer (AxL) und Eurocontrol Central Flow Management Unit (CFMU)
  im Rahmen des Projekts CNS FüZNatLV \emph{Führungszentrale Nationale Luftverteidigung},
  Uedem, Deutschland.
  Einführung und Wartung statischer Code-Analyse (Sonar) für die Erhebung von
  Qualitätsmetriken für Framework-, Server- und Client-Module.
  Unterstützung für das Schwesterprojekts \emph{Radar Recording} mit spezifischen Kenntnissen
  über das Eurocontrol ASTERIX Format bzw. dessen Datenkategorien und Java Netty
  Netzwerk-Bibliothek (UDP).
  \tech{Oracle Datenbank, Hibernate ORM, Netty, JBoss Application Server, SOAP}
}

\cventry
{2007--2011}
{Beratung, Architektur, Entwicklung}
{}
{Frequentis AG}
{Wien, Österreich}
{
  Neuentwurf und Umsetzung des JC3IEDM Datenspeichers, sowie der Schnittstelle
  zur DEM Protokollschicht für das \emph{Phönix Command and Control Information System (C2IS)}
  des Österreichischen Bundesheers.
  Modernisierung der Entwicklungswerkzeuge und -methoden; insbesondere Austausch von
  kommerziellen Lösungen durch Open-Source Alternativen.
  Einführen von Integrations- und Nightly Builds mit Continuous Integration Build Server (Hudson/Jenkins).
  \emph{Joint Consultation, Command and Control Information Exchange Data Model}
  (JC3IEDM) und \emph{Data Exchange Mechanism} (DEM) sind Spezifikationen des
  \emph{NATO Multilateral Interoperability Programme} (MIP).
  \tech{PostgreSQL, PL/pgSQL, Hibernate ORM, Eclipse RCP}
}

\cventry
{2004--2007}
{Teamleitung, Entwicklung}
{}
{Flughafen Wien AG}
{Schwechat, Österreich}
{
  Umsetzung von frachtspezifischen Anwendungen im Rahmen der Ablöse des
  Host-Systems durch Enterprise Java Plattform. Auswahl und Teamleitung für bis zu vier externe Entwickler.
  \tech{Oracle Datenbank, inkl. Advanced Queueing, WebSphere Application Server}
}

\cventry
{2002--2004}
{Teamleitung, Entwicklung}
{}
{Login GmbH}
{Wiener Neustadt, Österreich}
{
  Neuimplementierung eines bestehenden Dokumentenmanagementsystems auf Basis
  Enterprise Java. Die maßgeschneiderte Lösung ist spezialisiert auf die Verwaltung
  von domänenspezifischen Artefakte, die in der Anlagenplanung und im Anlagenbau der
  OMV/ÖMV, Schwechat, Österreich Verwendung finden. Teamleitung für drei Entwickler.
  \tech{Oracle Datenbank, Enterprise Java, JBoss Application Server}
}

\cventry
{1999-2002}
{Teamleitung, Entwicklung}
{}
{5 Point AG}
{Darmstadt, Deutschland}
{
  Umsetzung von verschiedenen geschäftskritischen Anwendungen (u.a. Händlerinformation,
  Gewährleistungsabwicklung) in Java und Enterprise Java auf IBM AS/400 für \u SkodaAuto Deutschland GmbH,
  Weiterstadt und Seat Deutschland GmbH, Mörfelden-Walldorf. Teamleitung und fachliche Führung
  von vier Entwicklern sowie technische Koordination mit Auftraggeber.
}


\section{Studentische Tätigkeiten}
\cventry
{1998}
{Entwicklung}
{}
{5 Point AG}
{Darmstadt, Deutschland}
{
  Entwicklung kleinerer Lösungen und Werkzeuge für \u SkodaAuto Deutschland GmbH,
  Weiterstadt, Deutschland.
}

\cventry
{1997}
{Entwicklung}
{}
{Hitech International Services GmbH - HISERV}
{Frankfurt/Main, Deutschland}
{
  IBM MQSeries Middleware Entwicklung in C/C++ für Hoechst AG, Frankfurt/Main und Alcatel, Frankreich.
}

\cventry
{1994--1996}
{Entwicklung}
{}
{Systan GmbH}
{Fränkisch-Crumbach, Deutschland}
{
  Entwicklung von einigen Lösungen und Werkzeugen in C und Borland Pascal für das
  Telekom D1 Abrechnungssystem DPPS (Data Post-processing System), DeTeMobil, Köln/Bonn, Deutschland.
}

\cventry
{1991--1993}
{Entwicklung}
{}
{CAP debis Systemhaus GmbH}
{Darmstadt, Deutschland}
{
  Entwicklung eines Editors für die Erstellung und Wartung von Teilen der Projektdokumentation
  (Eingabemasken der Terminal-Oberfläche) für das T-D1 DPPS Abrechnungssystems auf VAX/VMS mit
  VAX C/SQL und PostScript, sowie einer Schnittstelle zu DEC/CMS für das automatische Erstellen
  von Dokument-Revisionen.
}


\section{Open-Source Projekte}
\cventry
{2023}
{https://github.com/syncpoint/signal}
{}
{}
{}
{
  Synchrone JavaScript Signal-Bibliothek für reaktive funktionale Programmierung (FRP).
  Monadischer Datentyp, kompatibel mit Fantasy-Land Spezifikation und
  Transducer Protokoll.
}

\cventry
{2023}
{https://github.com/syncpoint/signs}
{}
{}
{}
{
  Neuimplementierung der \emph{milsymbol} Bibliothek von Måns Beckman mit Fokus auf
  Wartbarkeit und Erweiterbarkeit. Programmatisches Erzeugen von SVG-Symbolen für die
  Standards U.S. Department of Defense MIL-STD-2525 B bis D und NATO APP6-B und APP6-D.
}

\cventry
{2021}
{https://github.com/syncpoint/wkx}
{}
{}
{}
{
  Fork von https://github.com/cschwarz/wkx mit Fehlerkorrekturen und eigenem npm-Release.
  Encoder/Decoder für Geometrie-Formate \emph{Well-Know Text} und \emph{Well-Know Binary} (WKT/WKB).
}

\cventry
{2018--2025}
{https://github.com/syncpoint/ODINv2}
{}
{}
{}
{
  Geographische Lagedarstellung, insbesondere für militärische Zwecke. Die Symbolik
  basiert auf MIL-STD-2525C, U.S. Department of Defense. Electron Anwendung aufbauend auf
  React, OpenLayers und LevelDB. Unterstützung von macOS, Linux und Windows.
}


\section{Sprachen}
\cvitemwithcomment{Deutsch}{Muttersprache}{}
\cvitemwithcomment{Englisch}{fließend}{C2}
\cvitemwithcomment{Spanisch}{fortgeschritten}{B2}
\cvitemwithcomment{Japanisch}{grundlegende Grammatik und Schriftsysteme}{JLPT N5}

\section{Kenntnisse, Fähigkeiten}

\cvitem{Paradigmen}{
  Profundes Verständnis von objektorientiertem Softwareentwurf und --entwicklung.
  Praktische Erfahrungen mit lose gekoppelten, asynchronen Systemen im Kleinen wie im Großen.
  Gesundes Interesse an funktionaler Programmierung, nicht-blockierender Nebenläufigkeit
  sowie unveränderlichen Datenstrukturen und algebraische Datentypen.
}

\cvitem{Betriebs\-systeme}{macOS, Debian, RedHat, Windows, Android}
\cvitem{Programmier\-sprachen}{JavaScript/ECMAScript, PL/pgSQL, Scala, Java, bash/sh Script, \LaTeX}
\cvitem{Relationale Datenbanken}{PostgreSQL/PostGIS, SQLite, Oracle}
\cvitem{NoSQL}{LevelDB, MongoDB, Redis, CouchDB}
\cvitem{Middleware}{RabbitMQ, Matrix, ActiveMQ, {\O}MQ}
\cvitem{Frameworks}{React, Solid.js, OpenLayers, JavaScript Topology Suite}
\cvitem{Werkzeuge}{Node.js/npm, Electron, git, Docker, Visual Studio Code}

\section{Zertifikate}
\cvlistitem{IBM MQSeries Anwendungsprogrammierung auf Workstations}
\cvlistitem{Sun Certified Programmer for Java Platform}
\cvlistitem{Functional Programming Principles in Scala (mit Auszeichnung)}
\cvlistitem{Principles of Reactive Programming in Scala (mit Auszeichnung)}
\cvlistitem{MongoDB University -- MongoDB for Node.js Developers}

\section{Literatur}
\cvlistitem{\textbf{The Inmates Are Running the Asylum}: Why High Tech Products Drive Us Crazy and How to Restore the Sanity, \emph{Alan Cooper}}
\cvlistitem{\textbf{Living with Complexity} \emph{Donald A. Norman}}
\cvlistitem{\textbf{Lean Software Development}: An Agile Toolkit, \emph{Poppendieck/Poppendieck}}
\cvlistitem{\textbf{Domain Driven Design}: Tackling Complexity in the Heart of Software, \emph{Eric Evans}}
\cvlistitem{\textbf{Peopleware}: Productive Projects and Teams, \emph{Tom DeMarco, Timothy Lister}}
\cvlistitem{\textbf{Das Buch der Fünf Ringe} \emph{Miyamoto Musashi}}
\cvlistitem{\textbf{Die Kunst des Krieges} \emph{Sun Tzu/Sunzi}}
\cvlistitem{\textbf{Der Fürst} \emph{Niccolò Machiavelli}}

\section{Freizeit}
\cvlistitem{\textbf{Wandern, Laufen}}
\cvlistitem{\textbf{Shod\={o}}: Japanische Kalligraphie}
\cvlistitem{\textbf{Sh\={o}t\={o}kan Karate}: 1985-2022. Trainer, 2. Dan}
\cvlistitem{\textbf{Gitarre}: 1997-2004 Rhythmusgitarre in Coverband \emph{Over'n'Out}}

\end{document}
